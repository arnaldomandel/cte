\documentclass[brazil,11pt]{article}
\usepackage[brazil]{babel}
\usepackage[latin1]{inputenc}
\usepackage{amsmath}

\author{Karina Yaginuma \and Andr\'e Onuki dos Santos}
\title{Context Tree Estimation - Algorithm}

\begin{document}

\maketitle

The algorithm is divided in three parts: BIC Tree Calculation, Champion Tree Set Calculation and Bootstrapping.

The BIC Tree Calculation will be used many times to calculate the Champion Tree Set. Then this set is applied to the bootstrap procedure.

\section{BIC Tree Calculation}

Since this procedure is repeatedly called to search for the Champion Tree Set, some data which will alway be the same are pre calculated.

\begin{tabular}{rl}
Input: & Sample(s) ``$X^n_1$''\\
& the starting constant ``$c > 0$''\\
&maximum depth ``$d$''
\end{tabular}

\subsection{Pre Calculations}

For every string $w$ that:

\begin{equation}
l(w) \leq d
\end{equation}

\begin{equation}
\sum_{a \in A} Nn(wa) > 0
\end{equation}

Where $l(w)$ is the length of the string $w$ and Nn(w) is the number of times that $w$ occurs in the sample.

Calculate:
\begin{equation*}
L_w(X^n_1) = \prod_{a \in A} \hat{p}_n(a|w)^{N_n(wa)}
\end{equation*}
and
\begin{equation*}
df(w) = \sum_{a \in A} \chi(wa)
\end{equation*}


\subsection{Final Calculation}


\subsection{Implementation}

The Pre Calculations are implemented in the \emph{bic\_setup.c} file. The Final Calculation in the \emph{bic\_calculation.c} file.

To perform the calculations, data is saved in two trees. The first called Prob Tree holds the letters from the sample in the conventional way, meaning that a child node came in the sample after its parent node. The other three, called Bic Tree, holds letter reversed, the letter from a child node came before the one on its parent node.

Both trees are populated by reading every ``$d$'' sized word in the sample. So the filters described in ``Pre Calculations'' are already applied.

\section{Champion Tree Set Calculation}

\section{Bootstrapping}

\end{document}
