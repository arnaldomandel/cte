\documentclass[12pt]{article}
\usepackage{amsmath}
\usepackage{amsfonts}
\usepackage{amssymb}
\usepackage{latexsym}
\usepackage{xspace}

\newenvironment{demo}[1]{
  {\noindent\em #1\/}:}{\hfill$\Box$\vspace{3 mm}}
\newenvironment{proof}{
  {\noindent\em Proof\/}:}{\hfill$\Box$\vspace{3 mm}}
\newtheorem{lem}{Lemma}    % numeracao por seccao
\newtheorem{pro}[lem]{Proposition}
\newtheorem{cor}[lem]{Corollary}
\newtheorem{teo}[lem]{Theorem}

\newenvironment{muse}{\begin{quote}\small\sl}{\end{quote}}
\newenvironment{Muse}%
{\vspace{1em}%\begin{Sbox}
\begin{minipage}{12cm}
\rule{12cm}{1pt}\sffamily\noindent}%
{\newline\rule{12cm}{1pt}
\end{minipage}%\end{Sbox}\fbox{\TheSbox}
 \vspace{1em}}

%\newcommand{\forcemath}[1]{\ifmmode{#1}\else{$#1$\ }\fi}
\newcommand{\R}{\ensuremath{\mathbb{R}}\xspace}
\newcommand{\Q}{\ensuremath{\mathbb{Q}}\xspace}
\newcommand{\Z}{\ensuremath{\mathbb{Z}}\xspace}
\newcommand{\C}{\ensuremath{\mathbb{C}}\xspace}
\newcommand{\F}{\ensuremath{\mathbb{F}_2}\xspace}
\newcommand{\T}{\ensuremath{\mathcal{T}}\xspace}
\newcommand{\FF}{\ensuremath{\mathbb{F}_p}\xspace}
\newcommand{\si}{\ensuremath{\sigma}\xspace}
\newcommand{\mtiny}[1]{\mbox{\tiny\(#1\)}}
\newcommand{\dis}{\displaystyle}
\newcommand{\nid}{\noindent}
\newcommand{\bs}{\bigskip}
\newcommand{\conj}[2]{\ensuremath{\{#1\,|\;#2\}}}
\newcommand{\al}{\ensuremath{\alpha}\xspace}
\newcommand{\alinf}{\ensuremath{\alpha_{\inf{}}}\xspace}
\newcommand{\alsup}{\ensuremath{\alpha_{\sup{}}}\xspace}
\newcommand{\ccal}{\ensuremath{c_\alpha}\xspace}
\newcommand{\cd}{\ensuremath{(c,d)}\xspace}
\newcommand{\bip}{\prec\!\!\prec}
\newcommand{\lt}[2]{\ensuremath{#1\!\!\ltimes\!\!#2}}

\renewcommand{\theenumi}{\alph{enumi}}
\renewcommand{\labelenumi}{(\theenumi)}

\begin{document}
\title{Finding the list of champion trees}
\author{Arnaldo Mandel\thanks{Research partially supported by ...NeuroMat... MacLinc...}\\[-5pt]
  \emph{\small Departamento de Ci\^encia da Computa\c{c}\~ao, Universidade de
    S\~ao Paulo,}\\[-5pt]
  \emph{\small S\~ao Paulo, SP, Brazil 05508-970}\\[-5pt]
  \texttt{\small am@ime.usp.br}\\
  \textbf{\Large Working version, do not circulate}}

\maketitle
\section{TODO}
As it stands now, the paper is about generating a collection of subtrees of a
tree, maximizing a one parameter family of linear functions.  Progressively
more general contexts for the same question:
\begin{itemize}
    \item Same problem for maximum directed st-cuts in a digraph.  Maybe
  acyclic is a useful restriction.

    \item Same problem for vertices of a given polytope.
\end{itemize}

Some questions that should be settled:
\begin{itemize}
    \item What is the length of the champion list in the more general
  contexts.  Can it be exponential?  This is the most intriguing.  Already
  interesting: can it be larger than linear in case of an acyclic digraph?

    \item Can the responsibility conditions be lifted to directed (perhaps
  acyclic) digraphs so that the champion list will have the nested structure?
  This can beespecially relevant if there turns out that in general the list
  is long.
\end{itemize}

\section{Introduction}
State the general problem in a LP context.  State the problem of finding a
sequence of max st-cuts.  Restrict to contracted trees; the responsibility
condition.

Most of what comes here depends on the results of the TODO part.

Some references of note:

[AOS] Galves et al.

[Pr] Provan enumeration-vertices-network

[PS] Provan-Shier listing-directed-cuts

[CT] Csizar- Talata

Some general notation:

If \(f:X\rightarrow \R\) is a function, we extend it to subsets of \(X\) by
\(f(S)=\sum_{x\in S}f(x)\).  In any formulas involving sets, we omit curly
brackets from singletons; for instance, we write \(X\backslash v\) for
\(X\backslash \{v\}\).


\section{Champion trees}\label{sec:champion-trees}

Given a finite alphabet \(A\), let \(A^*\) denote the set of all words over
\(A\).  If \(w\in A^*\) can be written as concatenation \(w=xy\), then \(x\)
is a \emph{prefix} of \(w\) and \(y\) is a \emph{suffix} of \(w\); the latter
is denoted by \(y\prec w\) or \(w\succ y\).  We extend \(\prec\) to sets of
words by: \(S \prec T\) if for every \(s\in S\) there exists \(t\in T\) such
that \(s\prec t\).  Given a collection \(S\) of words closed under prefix, a
set \(C\) is a \emph{cut} (in [AOS] this is called a \emph{tree}) of \(S\)
provided it is a maximal \(\prec\)-antichain in \(S\).

In [AOS], the following setting occurs:

One is given a word \(X\) over a finite alphabet \(A\).  For each nonempty
word \(w\), let \(N_X(w)\) be the number of subwords (segments) of \(X\) equal
to \(w\) and \(N'_X(w)=\sum_{a\in A}N_X(wa)\). Clearly, \(N'_X(w)=N_X(w)\),
except for those \(w\) that are suffixes of \(X\); in those cases,
\(N'_X(w)=N_X(w)-1\).  The set of subwords of \(X\) equals \conj{w}{N_X(w)>0}.
Define now:
\begin{itemize}
  %   \item \(c(w) = \sum_{a\in A}N_X(wa)\log \frac{N_X(wa)}{N'_X(w)}\),
    \item \(c(w) = \sum_{a\in A}N_X(wa)\log \frac{N_X(wa)}{N'_X(w)}\),
    \item \(d(w) = |\conj{a\in A}{N_X(wa)>0}|\).
\end{itemize}

Given a real number \al, define, for a word \(w\), \(c_\al(w)=c(w)-\al d(w)\).

Now, a positive integer \(\ell\) is given.  Consider the set \(S_\ell(X)\) of
subwords of \(X\) of length at most \(\ell\).  A cut is \emph{\al-champion} if
it maximizes \(c_\al\) over all cuts of \(S_\ell(X)\), and it is minimal with
respect to this property.

A significant part of [AOS] depends on finding a list of cuts, so that for
every nonnegative \al\ an \al-champion cut is in the list.  In order to
prevent a degenerate case, we assume that \(N_X(p)>1\), where \(p\) is the
length \(\ell+1\) prefix of \(X\).

\begin{pro}\label{tau-good}
  Suppose that \(N_X(p)>1\), where \(p\) is the length \(\ell+1\)
  prefix of \(X\).  Then, for each nonempty subword \(w\) of \(X\) of length
  at most \(\ell\) :
  \begin{enumerate}
      \item Either \(d(w) < \sum_{a\in A}d(aw)\) or
      \item \(d(w) = \sum_{a\in A}d(aw)\) and \(c(w) \geq
    \sum_{a\in A}c(aw)\).
  \end{enumerate}
\end{pro}
\begin{proof}
  Clearly, \(\sum_{a\in A}d(aw)=|\conj{(a,b)}{N_X(awb)>0}|\geq d(w)\), since
  we suppose that every subword of length at most \(\ell+1\) occurs at least
  once with a preceding letter.  Moreover, for equality to hold, it is
  necessary that for each \(b\) such that \(N_X(wb)>0\) there exists a unique
  \(a=\varphi(b)\) such that \(N_X(\varphi(b)wa)>0\), hence,
  \(N_X(\varphi(b)wa)=N_X(wb)\), and for every letter
  \(\sigma\not=\varphi(b)\), \(N_X(\sigma wa)=0\).  Now we compute:
  \begin{eqnarray*}
    c(w) - \sum_{a\in A}c(aw)&=& c(w) -
    \sum_{a\in A}\sum_{b\in A}N_X(awb)\log\frac{N_X(awb)}{N'_X(aw)}\\
    &=& c(w) -
    \sum_{b\in A}N_X(\varphi(b)wb)\log\frac{N_X(\varphi(b)wb)}{N'_X(\varphi(b)w)}\\
    &=& c(w) -
    \sum_{b\in A}N_X(wb)\log\frac{N_X(wb)}{N'_X(\varphi(b)w)}
  \end{eqnarray*}
  Now, divide both sides by \(N'_X(w)\), and denote, for each \(a\in A\),
  \(p(a)=\frac{N_X(wa)}{N'_X(w)}\),
  \(q(a)=\frac{N_X(\varphi(a)wa)}{N'_X(\varphi(a)w)}\); notice that both \(p\)
  and \(q\) are probability distributions.  Then the right hand side turns
  into:
  \[
  \sum_{a\in A}p(a)\log\frac{p(a)}{q(a)},
  \]
  the Kullback-Leibler divergence \(D_{KL}(p||q)\) which everybody knows is
  nonnegative (reference... http://en.wikipedia.org/wiki/Gibbs'\_inequality).
\end{proof}




\section{Cuts in a tree}\label{sec:cuts-tree}
\begin{Muse}
The presentation here will probably be completely
rewritten, depending on the results of pending research  
\end{Muse}

Fix a finite rooted tree \T; we also use \T for its vertex set.  For
\(v,w\in\T\), denote \(v\prec w\) if \(v\) is in the path from the root to
\(w\) (possibly \(v=w\)); notice that \(\prec\) is a partial order, and \T is
its Hasse diagram.  We extend this order to subsets of \T: \(X\prec Y\) if for
every \(x\in X\) there is a \(y\in Y\) such that \(x\prec y\).  We also write
\(X\bip Y\) to mean \(X\prec Y\) and \(X\cap Y=\emptyset\).  A set
\(X\subseteq\T\) is a \emph{cut} if it is a minimal set separating the root
from the leaves; equivalently, it is a maximal antichain not containing
leaves, so any path from the root to a leaf contains a single member of \(X\),
and any such element is in a path to a leaf.  It follows easily that:

\begin{pro}
  For cuts \(X\) and \(Y\),  \(X\prec Y\) if and only if for every
  \(y\in Y\) there exists an \(x\in X\) such that \(x\prec y\).
\end{pro}

Let \(H\prec X\) be antichains.  Define
\(\lt{X}{H} = \conj{x \in
  X}{\text{there exists\ }h\in H\ \text{such that\ } h\prec x}\), and the
\emph{prunning} of \(X\) at \(H\) as \(X/H = X\backslash \lt{X}{H} \cup H\).
Clearly, if \(H\) is a cut, then \(H = X/ (H\backslash X)\), and the
Proposition above can be shortened to \(X\prec Y\) iff \(\lt{X}{H}=X\).

\begin{pro}
  Let \(X\) be a cut and \(H_1, H_2\) antichains such that
  \(H_1\prec H_2\prec X\). Then:
  \begin{enumerate}
      \item \((X/H_2)/H_1= X/(H_2/H_1)\)
      \item If \(\lt{H_2}{H_1}=H_2\), then \(X/H_2/H_1=X/H_1\).
  \end{enumerate}
\end{pro}

A \emph{weight} is a function \(f:\T\rightarrow\R\), and it is extended to
subsets of \T in the usual way.  Given a set \(X\) and a weight \(f\), we
define the real valued function \(f_X\) on subsets of \T by:
\(f_X(Y)=f(X)-f(Y)\).  It is immediate that:
\begin{itemize}
    \item \(f_X(Y)=-f_Y(X)\).
    \item For every \(X,Y_1,Y_2\), \(f_X(Y_1)-f_X(Y_2)= f(Y_2)-f(Y_1)\).
\end{itemize}

\begin{pro}\label{pro:prune}
  Let \(X\) be a cut, \(H\prec X\) an antichain and \(f\) a weight. Then:
  \begin{enumerate}
      \item \(f_X(X/H) = f(\lt{X}{H})-f(H)\).
      \item If \((H_1,H_2)\) is a partition of \(H\), then
      \[
        f_X(X/H)=f_X(X/H_1)+f_X(X/H_2).
      \]
      \item \(f_X(X/H) = \sum_{x\in H}f_X(X/x)\).
      \item If \(H\) is a cut then \(f_X(H) = f(\lt{X}{(H\backslash X)})-f(H\backslash X)\).
  \end{enumerate}
\end{pro}
\begin{proof}
  The first assertion follow directly from the definitions, and \((c)\)
  follows from \((b)\).  To get \((b)\), notice that
  \((\lt{X}{H_1},\lt{X}{H_2})\) is a partition of \(\lt{X}{H}\), and since
  \(f\) is additive on partitions, the result follows from \((a)\).  For
  \((d)\), notice first that \(H = X/H\); now consider the partition
  \((H\backslash X, H\cap X)\), and apply \((b)\).  Since \(X/(H\cap X) = X\),
  it follows that \(f_X(H\cap X)=0\).
\end{proof}

\begin{pro}\label{pro:telescope}
  Let \(X\) be a cut and \(f\) be a weight.  Then, for every \(w\bip X\),
  \[
    f_X(X/w) = f(\lt{X}{w}) - f(w) = \sum_{w\prec v\bip X}f(\T(v))-f(v).
  \]
\end{pro}
\begin{proof}
  The first equality is straight from the definition.  The second follows
  since the sum telescopes.
\end{proof}

In what follows, we consider a parametrized family of weights
\(\ccal=c+\al d\), where weights \(c,d\) are given and fixed, and \al is a
real parameter.  For a given \al, a cut \(X\) is \emph{\al-champion} if it
maximizes \ccal over all cuts, that is, \(\ccal(X)\geq\ccal(Y)\) for every cut
\(Y\); \(X\) is \emph{champion} if it is \al-champion for some \al.  The
objective here is, given \T and \ccal, to produce a collection of cuts such
that for every \al there is a minimal \al-champion cut in the list.

From the definition:

\begin{pro}\label{pro:dif}
  For all cuts \(X, Y\)
  \[
    \ccal(X) - \ccal(Y) = c_X(Y) - \al d_X(Y).
  \]
\end{pro}

Therefore:

\begin{teo}\label{teo:champ}
  A cut \(X\) is \al-champion if and only if, for every cut \(Y,\)
  \[
    \al d_X(Y) \leq c_X(Y).
  \]
\end{teo}

It follows that the set of \al values for which a given cut is \al-champion is
a closed interval (this also follows a priori from LP considerations).  It
will be useful to have a notation fo the extremes of such interval. Let
\(\alinf(X) = \inf\conj{\al}{X \text{ is \al-champion}}\), and define \alsup
similarly, noticing that \(\pm\infty\) are allowed.

\begin{cor}\label{cor:infsup}
  If X is a champion cut, then
  \begin{eqnarray*}
    \alinf(X) &=& \sup \left\{\frac{c_X(Y)}{d_X(Y)}\,|\,\text{cuts \(Y\) such that\ } d_X(Y)<0\right\},\\
    \alsup(X) &=& \inf \left\{\frac{c_X(Y)}{d_X(Y)}\,|\,\text{cuts \(Y\) such that\ } d_X(Y)>0\right\}.
  \end{eqnarray*}
\end{cor}

We will be interested in finding, for a given \al, a minimal (with respect to
\(\prec\)) \al-champion cut.  There is a bottom-up procedure for that was
already explored in [CT].  For a given node \(w\), let \(\T_w\) be the subtree
of \T rooted at \(w\), with \ccal suitably restricted.

\begin{pro}\label{pro:t-sub-w}
  If \(X\) is an \al-champion cut for \T and \(w\prec X\), then \(\lt{X}{w}\)
  is an \al-champion cut for \(\T_w\); if \(X\) is minimal, so is
  \(\lt{X}{w}\).
\end{pro}
\begin{proof}
  If \(Y\) is a cut of \(\T_w\), then \(Z=X\backslash\lt{X}{w}\cup Y\) is a
  cut, and \(\ccal(X)-\ccal(Z)=\ccal(\lt{X}{w})-\ccal(Y)\), so \(\lt{X}{w}\)
  is \al-champion for \(\T_w\).  If \(\lt{X}{w}\) is not minimal, there is a
  cut \(Y\) of \(\T_w\) such that \(\ccal(Y)=\ccal(\lt{X}{w})\); in that case,
  \(Z\) constructed as before would be \al-champion for \T, and \(Z\prec X\).
\end{proof}


At this point there is not more we can say in full generality; in particular,
it is not clear how a champion cut changes from \(X\) as \al increases just
past \(\alsup(X)\).  Moreover, the set of champion trees can be highly
unstructured...

\begin{Muse}
  I can produce examples in which the order structure of the champion set is
  quite arbitrary.  Indeed, I believe I can push the construction and show
  that basically all finite order types can be represented this way.
  Basic trick:
\end{Muse}

For that reason, in what follows we will stipulate a condition on weights.  We
will say that the pair \cd of weights is \emph{good} if for every internal
node \(v\)
\begin{enumerate}
    \item \(d(v)>0\) and,
    \item Either \(d(v) < d(\T(v)\) or \(d(v) = d(\T(v)\) and \(c(v) \geq c(\T(v)\)
\end{enumerate}
where \(\T(v)\) denotes the set of children of \(v\),

Proposition \ref{tau-good} states that the weights considered in section
\ref{sec:champion-trees} form a good pair.  Indeed, the notion of good pairs
came from trying to understandwhat really was going on in Lemma 10 of [AOS].

\begin{pro}\label{pro:good-lt}
  If \cd is good, then for every cut \(X\) and antichain \(H\bip X\), either
  \[
    d(\lt{X}{H}) > d(H)
  \]
  or
  \[
    d(\lt{X}{H}) = d(H) \;\text{and}\; c(H) \geq c(\lt{X}{H})
  \]
\end{pro}
\begin{proof}
  Since the sets \(\lt{X}{v} (v\in H)\) partition \(\lt{X}{H}\), it suffices to
  prove the case where \(H=\{w\}\).  From goodness and Prop.
  \ref{pro:telescope} applied to \(d\), it follows that either
  \(d(\lt{X}{H}) - d(H)>0\) or \(d(\lt{X}{v}) = d(v)\) or every
  \(w\prec v \bip X\).  In this case, from goodness and Prop.
  \ref{pro:telescope} applied to \(c\), we get the result.
\end{proof}

Restating the result above via Prop. \ref{pro:prune}(d):

\begin{pro}\label{pro:good-dif}
  If \cd is good, and \(Y\prec X\) are cuts, then \(d_X(Y)\geq 0\) and
  either \(d_X(Y)> 0\) or \(d_X(Y)= 0\) and \(c_X(Y) \leq 0\).
\end{pro}

\begin{cor}\label{cor:root}
  If \cd is good, then, as \(\alpha\rightarrow \infty\), the root is an
  \al-champion cut.
\end{cor}
\begin{proof}
  Take \(Y\) to be just the root, and apply the result above. It follows that
  \(d_Y(X)\leq 0\) for every cut \(X\).  By Cor. \ref{cor:infsup},
  \(\alsup(Y)\) is unbounded.
\end{proof}

We proceed now to a characterization of minimal \al-champions.

\begin{pro}\label{pro:min}
  Suppose that \cd is good.  If \(X\) is a minimal \al-champion cut, then, for
  every cut \(Y\prec X\), \(Y\neq X\), we have \(d_x(Y)>0\) and
  \(\al < \frac{c_X(Y)}{d_X(Y)}\).
\end{pro}
\begin{proof}
  From Prop. \ref{pro:good-dif}, if \(d_x(Y)=0\), \(c_X(Y)\leq 0\), and since
  \(X\) is \al-champion, Theorem \ref{teo:champ} implies that \(c_X(Y) = 0\),
  and \(Y\) is \al-champion.  So, if \(X\) is minimal, \(d_x(Y)>0\) and
  \(\ccal(X)> \ccal(Y)\).  The result now follows from Prop.  \ref{pro:dif}.
\end{proof}

\begin{teo}\label{teo:chain}
  Let \(X_1,X_2\) be cuts, and let \(\al_1<\al_2\in\R\). Suppose that \(X_1\)
  is \(\al_1\)-champion and \(X_2\) is \textbf{minimal} \(\al_2\)-champion.
  Then \(X_2\prec X_1\).
\end{teo}
\begin{proof}
  If not, there exists a \(w\in X_1\backslash X_2\), such that \(w\bip X_2\).
  Consider now \(\T_w\). Since \(\lt{X}{w}= w\), and this is
  \(\al_1\)-champion in \(\T_w\), by Prop. \ref{pro:t-sub-w}, we have
  \[
    c_{\al_1}(\lt{X_1}{w}) = c(w) - \al_1d(w) \geq c(\lt{X_2}{w}) - \al_1 d(\lt{X_2}{w}).
  \]
  Since \(X_2\) is \(\al_2\)-champion,
  \[
      c(\lt{X_2}{w}) - \al_2 d(\lt{X_2}{w}) \geq c(w) - \al_2d(w).
  \]
  Adding the two inequalities and noting that \(\al_2-\al_1 > 0\) leads to
  \[
    d(w) \geq d(\lt{X_2}{w}).
  \]
  By Prop. \ref{pro:good-lt}, \(d(w) = d(\lt{X_2}{w})\), and by
  Prop. \ref{pro:min} this contradicts the minimality of \lt{X_2}{w}, hence of
  \(X_2\) by Prop. \ref{pro:t-sub-w}.
\end{proof}

An immediate consequence of this is of note:

\begin{teo}\label{teo:total}
  If \cd is good, then there exists a family \(\mathcal{C}\) of cuts such
  that:
  \begin{enumerate}
      \item For every real \(\al\), \(\mathcal{C}\) contains a minimal
  \al-cut,
      \item \(\mathcal{C}\) is totally ordered by \(\prec\).
  \end{enumerate}
  \((a)\) For every real \(\al\), \(\mathcal{C}\) contains a minimal
\end{teo}

It remains to show how to get from one cut to the next.

\begin{teo}\label{teo:prune}
  Suppose that the cut \(X\) is minimal \al-champion.  Let \(H\bip X\) be an
  antichain and let \((H_1, H_2)\) partition \(H\).  If \(X/H\) is champion,
  then, either
  \[
    \frac{c_X(X/H_1)}{d_X(X/H_1)} = \frac{c_X(X/H_2)}{d_X(X/H_2)} = \frac{c_X(X/H)}{d_X(X/H)}
  \]
  or, for \(i=1\) or \(i=2\),
  \[
    \al < \frac{c_X(X/H_i)}{d_X(X/H_i)} < \alinf(X/H). 
  \]
\end{teo}
\begin{proof}
  From the minimality of \(X\) and Prop. \ref{pro:min}, it follows that all
  denominators are positive.  Appling Prop. \ref{pro:prune}\((b)\) to \(c\)
  and \(d\), it follows that if two of the fractions above are equal, so is
  the third.

  If \(\frac{c_X(X/H)}{d_X(X/H)}<\frac{c_X(X/H_1)}{d_X(X/H_1)}\) and
  \(\frac{c_X(X/H)}{d_X(X/H)}<\frac{c_X(X/H_2)}{d_X(X/H_2)}\), clearing
  denominators and adding yields a contradiction.  Hence, for some \(i\),
  \(\frac{c_X(X/H)}{d_X(X/H)}>\frac{c_X(X/H_1)}{d_X(X/H_1)}\).  But
  \(\frac{c_X(X/H)}{d_X(X/H)}= \frac{c_{X/H}(X)}{d_{X/H}(X)}\) and
  \(d_{X/H}(X)=-d_X(X/H)<0\), hence, by Cor. \ref{cor:infsup},
  \(\alinf(X/H)\geq \frac{c_X(X/H)}{d_X(X/H)}\), and we get the desired upper
  bound.  The lower bound comes out from Prop. \ref{pro:min} and the fact that
  \(X/H_i \prec X\).
\end{proof}

The following is the key for an efficient computation of the sequence of
champion cuts:
\begin{teo}\label{teo:next}
  Let \(X\) be a minimal \al-champion cut, where \cd is a good pair.  Let
  \(w\bip X\) minimize \(\beta=\frac{c_X(X/w)}{d_X(X/w)}\).  Then,
    \(\alsup(X)=\beta\) and \(X/w\) is \(\beta\)-champion.
\end{teo}
\begin{proof}
  Since \(X\) is not just the root, choose \(\gamma \geq \alsup(X)\), and let
  \(Y\) be a \(\gamma\)-champion cut.  By Theorem \ref{teo:chain},
  \(Y\prec X\), hence \(Y=X/H\) for some \(H\bip X\).  Repeated application of
  Theorem \ref{teo:prune} yields a \(v\in H\) such that
  \(\frac{c_X(X/v)}{d_X(X/v)}\leq \alinf(X/H)\).  The choice of \(w\) implies
  that \(\beta\leq\alinf(X/H)\leq\gamma\).  But, with \(\gamma=\alsup(X)\),
  this is a contradiction, unless \(H=\{w\}\), which yields the result.
\end{proof}
\end{document}
